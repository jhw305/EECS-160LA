In the feedback system displayed in (\ref{fig:fig_2}), an input (1/s) is applied, and so $V_{out}$(s) takes the raw form: (\ref{eq:feedback_1}).

\begin{figure}[h!]
	\centering
	\caption{Modified system with feedback gain}
	\label{fig:fig_2}
\end{figure}

\begin{equation}
	\label{eq:feedback_1}
	V_{out}(s) = \frac{1}{s}\frac{\frac{k C_{1}}{C_{2}s + 1}}{1 +\frac{0.1k C_{1}}{C_{2} s + 1}}
\end{equation}

With some manipulation, this turns into (\ref{eq:feedback_2}).

\begin{equation}
    \label{eq:feedback_2}
    V_{out}(s) = \frac{k C_{1}/C_{2}}{s}\frac{1}{s+\frac{1+0.1k C_{1}}{C_{2}}}
\end{equation}

With an inverse Laplace transform, this turns into (\ref{eq:feedback_3}).

\begin{equation}
    \label{eq:feedback_3}
    V_{out}(t) = \frac{k C_{1}/C_{2}}{\frac{1+0.1k C_{1}}{C_{2}}}(1-\mathrm{e}^{-(\frac{1+0.1 k C_{1}}{C2})t})
\end{equation}

Finally, this simplifies to (\ref{eq:feedback_4}).

\begin{equation}
    \label{eq:feedback_4}
    V_{out}(t) = \frac{k C_{1}}{1+0.1k C_{1}}(1-\mathrm{e}^{-(\frac{1+0.1 k C_{1}}{C2})t})
\end{equation}

This implies that the steady state value is $\frac{k C_{1}}{1+0.1k C_{1}}$, and the time constant is $\frac{C2}{1+0.1 k C_{1}}$.
